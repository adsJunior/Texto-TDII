% Nome do capítulo
\chapter{Introdução}
% Label para referenciar
\label{cap:1}

% Diminuir espaçamento entre título e texto
\vspace{-1.9cm}
% Expressivo crescimento??? 
Atualmente há uma produção massiva de conteúdo visual, como imagens e vídeos, disponíveis para utilização. Podemos vincular este fato à popularização das tecnologias produtoras destes tipos de conteúdo, como celulares com câmeras, bem como a expansão da Internet e seus canais de comunicação, que utilizam de imagens e vídeos para divulgação de informações. As informações contidas em imagens e vídeos são utilizadas pelas mais variadas áreas da sociedade, como médica, industrial e, até mesmo, para fins pessoais e de entretenimento.

O processamento digital de imagens contribui para a descoberta de informação visual contida em imagens e vídeos. Um dos problemas de processamento digital de imagens amplamente explorados na literatura é o problema de classificação e localização de objetos em imagens. \citeonline{everingham-2015} definem que o problema de classificação consiste em responder para cada classe de objeto se existe ou não uma ou mais instâncias daquele objeto na imagem e, o problema de localização consiste em dizer onde na imagem estão as instâncias dos objetos reconhecidos pelo classificador.

A Figura \ref{fig:exemploclassdet} mostra exemplos de como devem ser as saídas de um algoritmo de classificação e localização de objetos. Os retângulos destacados em torno dos objetos são resultados do algoritmo de localização que determina que dentro daquela região existe um objeto de interesse e os rótulos destacados em cima dos retângulos são os resultados do algoritmo de classificação, que determina que o objeto contido dentro da região pertence àquela classe.

  \begin{figure}[t!]
% Alterar espaçamentos antes e depois do caption
  \setlength{\abovecaptionskip}{0pt}
  \setlength{\belowcaptionskip}{0pt}
% Caption
  \caption[Exemplo de Classificação e Localização]{Exemplo de Classificação e Localização de Objetos}
  \centering
  \includegraphics[width=.9\textwidth]{imagem/0x_classdetec.jpg}
% Caption centralizada
  \captionsetup{justification=centering}
  \captionfont{\small{\textbf{\\Fonte: \citeonline{everingham-2015}.}}}	
  \label{fig:exemploclassdet}
  \end{figure}

O uso das \ac{CNN} tem se tornado cada vez mais populares, principalmente em alguns dos problemas clássicos de processamento de imagens. A primeira abordagem desse tipo foi proposta por \citeonline{fukushima-1980} para fazer reconhecimento de caracteres escritos a mão. Mais tarde, \citeonline{lecun-1998} desenvolveram uma arquitetura de \ac{CNN} para a mesma tarefa e obtiveram uma acurácia de $70\%$, sendo essa muito superior aos resultados obtidos por qualquer outro classificador utilizado até então.

%%% p1 Escrever sobre os resultados obtidos pelas CNNs em classificação de imagens
Contudo, as \ac{CNN} só passaram a ser utilizadas com mais frequência algum tempo depois. \citeonline{deng-2009} lançaram o ImageNet, uma base de dados de larga escala com mais de 14 milhões de imagens anotadas manualmente divididas em mais de 22 mil classes diferentes. Essa base de imagens pode ser usada para as tarefas de classificação, detecção e localização de objetos. Além disso, \citeonline{deng-2009} também lançaram o \ac{ILSVRC}, uma competição global anual, aonde os competidores são avaliados nas tarefas de classificação de imagens, detecção de objetos e localização de objetos. \citeonline{krizhevsky-2012} alcançaram o melhor resultado no \ac{ILSVRC} de 2012 usando a AlexNet - uma \ac{CNN} - e desde então, as \ac{CNN}s obtém sempre os melhores resultados nos desafios.

Além da classificação de imagens, as \ac{CNN} têm sido frequentemente usadas para resolver problemas de localização e Classificação \cite{redmon-2015,wei-2015,sren-2017,cheng-2017}, segmentação semântica \cite{long-2014, noh:2015} e segmentação de objeto em vídeos \cite{caelles-2017, voigtlaender-2017}.

\section{Motivação}
\label{secao:1:1}

%%% p1 Mencionar como as CNNs tem sido objetos de estudo recentemente
Estudos envolvendo \ac{CNN}s foram realizados recentemente em áreas como classificação de imagens \cite{krizhevsky-2012, simonyan-2014}, classificação e detecção de objetos \cite{cheng-2017, lin-2014}, segmentação semântica de imagens \cite{long-2014, noh:2015}, segmentação de objetos em vídeos \cite{caelles-2017, voigtlaender-2017}. Isso se deve aos bons resultados obtidos pelos métodos aplicados nas respectivas áreas.

%%% p2 Mencionar como as aplicações dos algoritmos de detecção
Além disso, o problema de classificação e detecção de objetos é aproveitado como solução para problemas mais específicos, como detecção facial \cite{yang-2018}, contagem de pessoas \cite{pren-2017} e detecção de pedestres \cite{lan-2018}. 

%%% p3 Mencionar a espectativa ao usar deconvolução
Por fim o uso de camadas adicionais de convolução ao final de uma \ac{CNN} pode gerar bons resultados na localização e classificação de objetos como apresentado por \citeonline{wei-2015} e \citeonline{cheng-2017}. Em suma, pode-se dizer que o tema ainda tem muito que ser explorado, que a proposta é promissora, e que bons resultados podem trazer contribuições para trabalhos futuros.

%Por fim o uso das camadas de deconvolução no método proposto traz boas expectativas, uma vez que já há resultados na literatura que mostram a sua eficiência \cite{noh:2015, cheng-2017}. Em suma, pode-se dizer que o tema ainda tem muito que ser explorado, que a proposta é promissora, e que bons resultados podem trazer contribuições para trabalhos futuros.

\section{Objetivos}
\label{secao:1:2}  
% Reescrever o Objetivo geral e os objetivos específicos

O objetivo do trabalho é implementar modificações na arquitetura da \ac{DenseNet}121 de forma a possibilitar que ela faça localização e classificação de objetos, conforme proposto por \citeonline{wei-2015} e por \citeonline{liu-2017}.

\subsection{Objetivos Específicos}

Para que o objetivo geral do trabalho seja alcançado, será necessário cumprir os seguintes objetivos:

\begin{itemize}
	\item Fazer um estudo para entender o funcionamento dos algoritmos de Localização e classificação de objetos;
	\item Levantar quais bases de dados serão utilizadas para fazer o treinamento e a avaliação do modelo proposto. Além disso, também levantar qual ou quais métricas serão utilizadas para avaliar os resultados;
	\item Implementar uma Rede Neural Convolutiva Densa modificada com os módulos de predição para localização e classificação propostos por \citeonline{wei-2015} e \citeonline{liu-2017};
	\item Fazer o treinamento do modelo desenvolvido com as bases selecionadas. Após o treinamento, fazer os testes e registrar os resultados;
	\item Analisar os resultados, comparando-os com a literatura.
\end{itemize}

\section{Justificativa}
\label{secao:1:3}

Com os avanços tecnológicos alcançados ao longo das últimas décadas, a produção de conteúdo multimídia tem crescido consideravelmente e tem sido amplamente explorada pelos mais diversos setores da sociedade. Todos os dias mais de 95 milhões de fotos e vídeos são postadas no Instagram\footnote{https://www.instagram.com}. Esse imenso volume de dados que são produzidos e acessados em multimídia inviabiliza a manipulação deste conteúdo por meio de ação humana; criando assim, a necessidade de automatizar a recuperação e análise de informações relevantes contidas nas mesmas.

Nesse sentido, todos os dias são estudados novos algoritmos e novas técnicas para fazer recuperação de informação multimídia. E uma das formas amplamente utilizadas é a de localização e classificação de objetos em imagens. \citeonline{everingham-2015} definem que o problema de classificação consiste em responder para cada classe de objeto se existe ou não uma ou mais instâncias daquele objeto na imagem e, o problema de localização consiste em dizer onde na imagem estão as instâncias dos objetos reconhecidos pelo classificador.

É válido estabelecer que ``[...]os mecanismos sensitivos dos seres humanos (como visão e audição) sugerem a necessidade de uma arquitetura profunda para extrair a sua estrutura complexa.'' \cite{deng-2014}. Porém, de acordo com \citeonline{caelles-2017}, uma grande limitação ao utilizar arquiteturas de \textit{Deep Learning} é a necessidade de treinamento com um grande volume de dados. Essa limitação torna o processo mais custoso em termos de tempo de processamento e outros recursos computacionais(como memória). Além disso, a base de dados deve ter os resultados da classificação e localização esperados feitos de forma manual, o que aumenta o esforço humano.

Tendo isso em vista, a proposta de utilizar a \ac{DenseNet} \cite{liu-2017} pré-treinada em classificação de imagens visa diminuir o custo por meio do \textit{Transfer Learning}. Além disso, utilizando as camadas adicionais de convolução como proposto por \citeonline{wei-2015}, espera-se obter resultados ainda melhores. Esses resultados melhores repercutiriam de forma positiva em outras áreas relacionadas à localização e classificação de objetos. 

\section{Organização do texto}
\label{secao:1:4}

O Capítulo \ref{cap:2} traz os principais conceitos relacionados à \textit{Deep Learning}, redes neurais artificiais, classificação de objetos em imagens e detecção de objetos em imagens, necessários à compreensão do trabalho. O Capítulo \ref{cap:3} contém um resumo dos trabalhos relacionados à monografia. O Capítulo \ref{cap:4} contém a metodologia a ser seguida para o desenvolvimento do trabalho. O Capítulo \ref{cap:5} descreve os experimentos realizados e os resultados obtidos. O Capítulo \ref{cap:6} apresenta a conclusão do trabalho com base nos resultados obtidos e apresenta propostas de trabalhos futuros.