% Nome do capítulo
\chapter{Proposta Técnica}
% Label para referenciar
\label{cap:4}
% Diminuir espaçamento entre título e texto
\vspace{-1.9cm}


\section{Metodologia}
\label{secao:4:1}

A seguir, serão definidos os passos necessários para a elaboração do trabalho proposto.

\subsection{Levantamento Bibliográfico}
\label{subsecao:4:1:1}

Nesta fase será feito um levantamento na literatura de tudo o que é julgado necessário para realizar o trabalho proposto. Será feito um estudo para compreensão da arquitetura \ac{DenseNet} \cite{liu-2017}. Além disso, serão feitos estudos com métodos da literatura como \ac{YOLO} \cite{redmon-2015}, \ac{SSD} \cite{wei-2015} e \ac{DSSD} \cite{cheng-2017}. Serão feitos também estudos para analisar o conteúdo da base de dados para testes. A base utilizada será a \ac{PASCAL VOC} \cite{everingham-2015}. Além disso, nessa etapa serão levantadas as formas de avaliar os resultados dos algoritmos.

\subsection{Testes e avaliações com métodos da literatura}
\label{subsecao:4:1:2}

Nesta etapa serão feitos os testes com os métodos já implementados na literatura. O objetivo é compreender o funcionamento, levantar os principais obstáculos nas respectivas implementações e selecionar as melhores tecnologias para a realização do projeto proposto. Nesta etapa também serão testados os algoritmos para avaliar os resultados obtidos, propostos por \citeonline{everingham-2015} e por \citeonline{lin-2014}.

\subsection{Desenvolvimento do protótipo inicial}
\label{subsecao:4:1:3}

A proposta é desenvolver um protótipo inicial usando a arquitetura da \ac{DenseNet} 121 apresentada por \citeonline{liu-2017} com alterações. As alterações a ser feitas são a remoção da camada final de classificação e a inserção de camadas intermediárias de convolução para realizar a localização dos objetos em diferentes escalas, como apresentado por \citeonline{wei-2015}. Além disso, serão aplicadas as mesmas modificações aplicadas inicialmente na \ac{ResNet} 101 para gerar o primeiro modelo proposto por \cite{cheng-2017}.

\subsection{Desenvolvimento da arquitetura final usando deconvolução}
\label{subsecao:4:1:4}

Depois de alterar a arquitetura para fazer a detecção em múltiplas escalas usando convolução, será feita uma nova modificação da arquitetura, acrescentando camadas de deconvolução. Para cada camada de deconvolução é acrescentado um módulo de predição, que fará a localização e a classificação dos objetos. O acréscimo das camadas de deconvolução e dos novos módulos de predição seguem o modelo proposto por \citeonline{cheng-2017}.

\subsection{Avaliação dos resultados e comparação com o Estado da Arte}
\label{subsecao:4:1:5}

Por fim, erá feita a avaliação dos resultados dos métodos propostos e a comparação dos mesmos com os resultados do estado da arte. Essas avaliações serão feitas com base nas métricas que serão apresentadas na Sub-seção \ref{subsecao:4:1:1}. O objetivo é dizer o quão efetivo foi a utilização de uma arquitetura de \ac{CNN} mais moderna e avançada com métodos já conhecidos na resolução do problema de localzação e classificação.

\subsection{Monografia}
\label{subsecao:4:1:6}

Por fim será confeccionado uma monografia apresentando e descrevendo a implementação do método proposto, os resultados obtidos, uma avaliação dos resultados e uma comparação com o estado da arte.

\section{Cronograma}
\label{secao:4:2}

O projeto será desenvolvido ao longo do ano 2019 seguindo o cronograma apresentado na Tabela \ref{tab:tabela1}. O período está dividido em bimestres, sendo o primeiro bimestre equivalente aos meses de janeiro e fevereiro de 2019, e assim sucessivamente.

\begin{table}[H]
	\centering
	% 	\vspace{-0.3cm} % espaço entre titulo e tabela
	\caption[Cronograma de desenvolvimento]{Cronograma de desenvolvimento do projeto}
	\label{tab:tabela1}
	\begin{tabular}{|c|c|c|c|c|c|c|}
		\hline
		\multirow{2}{*}{Atividades}            & \multicolumn{6}{c|}{Bimestre} \\ \cline{2-7} 
		& 1   & 2   & 3  & 4  & 5  & 6  \\ \hline
		Levantamento Bibliográfico             & x   & x   & x  &    &    &    \\ \hline
		Testes com métodos da literatura       &     &     & x  & x  &    &    \\ \hline
		Desenvolvimento da arquitetura inicial &     &     & x  & x  &    &    \\ \hline
		Desenvolvimento da arquitetura final   &     &     &    & x  & x  &    \\ \hline
		Avaliação dos resultados               &     &     &    &    & x  & x  \\ \hline
		Elaborar Monografia                    &     &     &    &    & x  & x  \\ \hline
	\end{tabular}
	\vspace{0.1cm} 
	{\footnotesize\\ \textbf{Fonte: Elaborado pelo autor}}
\end{table}


% Usar espaço para descrever base de testes, método de avaliação e fluxograma.


