% Resumo
\begin{resumo}
% Diminuir espaçamento entre título e texto
\vspace{-1cm}

% Texto do resumo: sem paragrafo, justificado, com espaçamento 1,5 cm
\onehalfspacing

\noindent 
   Atualmente há um expressivo crescimento no volume de conteúdo visual, como imagens e vídeos, disponíveis para utilização. As informações contidas em imagens e vídeos são utilizadas pelas mais variadas áreas da sociedade, como médica, industrial e, até mesmo, para fins pessoais. Todos os dias são estudados novos métodos computacionais para fazer recuperação e análise de imagens. Dois métodos que tem sido frequentemente estudados em conjunto são a localização e classificação de objetos. Arquiteturas de \textit{Deep Learning} podem ser implementadas para fazer localização e classificação de objetos, obtendo bons resultados, porém possuem certas limitações como o tamanho dos objetos ou a resolução da imagem. Para contornar essa limitação, pretende-se acrescentar camadas de convolução ao final para fazer a localização dos objetos de diversos tamanhos de forma mais apurada. Após as camadas finais de Convolução, serão acrescentadas camadas de Deconvolução, para aumentar a escala dos mapas, e decodificar os resultados das convoluções de forma a aumentar  precisão do resultado.

% Espaçamento para as palavras-chave
\vspace*{.75cm}

% Palavras-chave: sem parágrafo, alinhado à esquerda
\noindent Palavras-chave: \textit{Deep Learning}, Localização, Classificação, Convolução, Deconvolução.\\
% Segunda linha de palavras-chave, com espaçamento.
%\indent\hspace{2cm}Palavra.

\end{resumo}